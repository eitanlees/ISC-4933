\section{Introduction to Linear Systems}

A linear system is a set of equations of the form 
%
\begin{equation*}
    \sum_{j=1}^n a_{ij} x_j  = b_i \qquad i=1, \ldots, m
\end{equation*}
%
We are typically given $a_{ij}$ and one of $x_j$ or $b_i$. 
Our job is to find the other. In matrix form, 
\begin{equation*}
    A \Vec{x} = \Vec{b}
\end{equation*}
where 
$A \in \mathbb{R}^{m \times n}$, 
$\Vec{x}\in \mathbb{R}^{n}$, 
$\Vec{b}\in \mathbb{R}^{m}$
%
\begin{equation*}
    A = 
    \begin{bmatrix}
    a_{11} & a_{12} & \ldots & a_{1n}\\
    a_{21} & a_{22} & \ldots & a_{2n}\\
    \vdots & \vdots  & \ddots & \vdots\\
    a_{m1} & a_{m2} & \ldots & a_{mn}\\
    \end{bmatrix}
    ,\quad
    \Vec{x}  = \begin{bmatrix}x_1\\\vdots\\x_n\end{bmatrix}
    ,\quad
    \Vec{b}  = \begin{bmatrix}b_1\\\vdots\\b_m\end{bmatrix}
\end{equation*}

Linear systems arise when we 
\begin{itemize}[label={--}]
    \item Discretize a linear differential equation or integral equation.
    \item Linearizing a nonlinear differential equation 
	    or a nonlinear integral equation.
    \item Interpolating data with basis functions 
	    such as polynomials, wavelets, Fourier expansions.
    \item Optimizing an objective function $f\left(\Vec{x}\right)$. 
    \item Statistical operations such as lienar regression. 
    \item Solve an inverse problem. 
    \item Data sciences.
\end{itemize}

With $A\in\mathbb{R}^{m\times n}$, we have 3 cases

\begin{center}
	\begin{enumerate}[1)]
		\item $m<n$
		\item $m>n$
		\item $m=n$
	\end{enumerate}
\end{center}

If $m<n$, we have fewer equations than unknowns. 
Such a system typically has infinitely many solutions, 
and such a system is said to be \emph{underdetermined}. 
It can also have no solutions. 

\underline{EX}:
\begin{align*}
    x_1 + x_2 + 2x_3 &= 0 & x_1 + x_2 - 2x_3 &= 0\\
    x_1 - 2x_2 + 3x_3 &= 1 & 2x_1 +2x_2 - 4x_3&=7\\
    \text{\textcolor{red}{infinitely many }}&\text{\textcolor{red}{solutions}}
    & \text{\textcolor{red}{no solutions}}
\end{align*}

Underdetermined problems arise, for example, in inverse problems. 
Here, there is typically some kind of field that is unknown, 
and we only have a few measurements. 
For example in seismic, the distribution of the formations underground 
are predicted from only a few available measurements. 

If we have $m>n$, we have more equations than unknowns. 
Such a system typically has no solution, 
and such a system is said to be \emph{overdetermined}. 
It can also have 1 solution or infinitely many solutions. 


\underline{EX}:
\begin{align*}
    x+y&=1 & x+y&=1 & x+y&=1 \\
    2x+2y&=2 & 2x+2y&=2 & x-y&=2 \\
    3x+3y&=3 & 3x+4y&=3 & 3x-y&=4 \\
    \text{\textcolor{red}{infinitely many }}&\text{\textcolor{red}{solutions}}
    & &\hspace{-3em}\text{\textcolor{red}{one solutions}}
    & &\hspace{-3em}\text{\textcolor{red}{no solutions}}
\end{align*}

Overdetermined problems arise, for example, in interpolation. 
Suppose we are given data points $(x_i, y_i),\, i=1, \ldots, m$ 
and we seek a linear function interpolating this data.  
Suppose the interpolant is $f(x) = c_0 + c_1 x$
%
\begin{align*}
    c_0 + c_1x_1 &= y_1 \\
    c_0 + c_1x_2 &= y_2 \\
    \vdots\\
    c_0+c_1x_m &= y_m
\end{align*}


\begin{center}
	\begin{tikzpicture}

	% Axis
	\draw[->] (-.2, 0) -- node[below]{x}(3, 0);
	\draw[->] (0, -.2) -- node[left]{y}(0, 3);

	% Random points
	\newcommand{\noise}{0.5}
        \pgfmathsetseed{7}
        \foreach \i in {0.1, 0.5, ..., 2.8} 
		\pgfmathsetmacro{\x}{\i + \noise*rand}
        	\pgfmathsetmacro{\y}{\i + \noise*rand}
		\draw [fill] (\x, \y) circle [radius=.5mm];
	
	% Line of best fit
	\draw[dashed, Gray] (0,0) -- (3,3);

\end{tikzpicture}

\end{center}

Let's focus on the case $m=n$ and suppose that
$A\Vec{x} = \Vec{b}$ has a unique solution for every $\Vec{b}$. 
The three main tasks to be done are 
%
\begin{enumerate}[1)]
    \item Compute $A\Vec{x}$ for some $\Vec{x}\in\mathbb{R}^n$
    \item Solve $A\Vec{x}=\Vec{b}$ for some $\Vec{b}\in\mathbb{R}^n$
    \item Decompose $A$ as 
    
	    \begin{tabular}{lp{6cm}}
         $A=LU$ & Lower-Upper Decomposition  \\
         $A=QR$ & where Q is orthogonal ($Q^T=Q^{-1}$) and R is upper triangular\\
         $A=U\Sigma V^T$ & Singular Value Decomposition 
	 			where $U, V$ are orthogonal and $\Sigma$ is diagonal\\
         $A=VDV^{-1}$ & Eigenvalue Decomposition\\
         $A=A_{(:, J)}X$ & Interpolative Decomposition
    \end{tabular}
\end{enumerate}
